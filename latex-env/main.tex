\listfiles
\documentclass[manuscript, review, screen]{acmart}
%\setcitestyle{super,sort&compress}
\citestyle{acmauthoryear}
\usepackage{booktabs} % For formal tables
\usepackage{minted}
\usepackage{subcaption}
\usepackage[ruled]{algorithm2e} % For algorithms
\usepackage{graphicx}  % to insert an image
\graphicspath{ {./images/} }

\newcommand{\todo}[1]{{\color{red}\bfseries [[#1]]}}

% Metadata Information
% \acmJournal{CACM}
% \acmVolume{9}
% \acmNumber{4}
% \acmArticle{39}
 \acmYear{2018}
% \acmMonth{3}

%\acmBadgeL[http://ctuning.org/ae/ppopp2016.html]{ae-logo}
%\acmBadgeR[http://ctuning.org/ae/ppopp2016.html]{ae-logo}

% Copyright
\setcopyright{acmcopyright}
%\setcopyright{acmlicensed}
%\setcopyright{rightsretained}
%\setcopyright{usgov}
%\setcopyright{usgovmixed}
%\setcopyright{cagov}
%\setcopyright{cagovmixed}

% DOI

\acmDOI{0000001.0000001}


% Document starts
\begin{document}
\title{Example Paper}
% \titlenote{This is a titlenote}
% \subtitle{This is a subtitle}
% \subtitlenote{Subtitle note}


\author{Chris Parnin}
%\orcid{1234-5678-9012-3456}
\affiliation{%
  \institution{North Carolina State University}
  \country{USA}}
\email{cjparnin@ncsu.edu}

\begin{abstract}
To be written 10 minutes before submission.
\end{abstract}


%
% The code below should be generated by the tool at
% http://dl.acm.org/ccs.cfm
% Please copy and paste the code instead of the example below.
%
\begin{CCSXML}
<ccs2012>
<concept>
<concept_id>10003456.10003457.10003458</concept_id>
<concept_desc>Social and professional topics~Computing industry</concept_desc>
<concept_significance>300</concept_significance>
</concept>
<concept>
<concept_id>10011007.10010940</concept_id>
<concept_desc>Software and its engineering~Software organization and properties</concept_desc>
<concept_significance>300</concept_significance>
</concept>
<concept>
<concept_id>10011007.10011074.10011111.10011697</concept_id>
<concept_desc>Software and its engineering~System administration</concept_desc>
<concept_significance>100</concept_significance>
</concept>
</ccs2012>
\end{CCSXML}

\ccsdesc[300]{Social and professional topics~Computing industry}
\ccsdesc[300]{Software and its engineering~Software organization and properties}
\ccsdesc[100]{Software and its engineering~System administration}
%
% End generated code
%

\keywords{continuous deployment, devops, practices}




\maketitle

\section{Introduction}

Predictable, rapid, and data-driven feature rollouts; lightening-fast and automated fix deployment{\textemdash} worldwide, companies striving for these and other benefits are transitioning toward the use of continuous deployment practices.  Continuous deployment is a software engineering process where incremental software changes are automatically tested and frequently deployed to production environments without manual steps in the deployment pipeline~\cite{Rahman:2015}. With continuous deployment, the elapsed time for a change made by a developer to reach a customer can be measured in days or even hours. Continuous deployment enables companies, such as Facebook \cite{Facebook_2016}, to make hundreds or thousands software changes to live computing infrastructure every day, while maintaining service to millions of customers.  Such ultra-fast changes create a new reality in software development.  

Continuous deployment is enabled through both a mastery in the creation of a \emph{scalable, resilient, reproducible, and secure infrastructure} and the navigation of organizational cultural changes caused by deployment speed.  Over the past three years, we have held the Continuous Deployment Summit, hosted at Facebook \cite{parnin2016top}, Netflix, and Google. Representatives from companies, including Cisco, Disney, Facebook, Google, IBM, LexisNexis, Microsoft, Netflix, SAS, Slack, and Twitter have shared how their triumphs and struggles of their transition to continuous deployment practices -- each year the companies press on, getting ever faster. In this paper, we peer under the covers into the common strategies and practices used by continuous deployment pioneers and adopted by newcomers as they transition and use continuous deployment practices at scale.  

\section{Code example}

\begin{figure}[t]
\begin{minipage}{\linewidth}
\begin{minted}[escapeinside=||,fontsize=\scriptsize,   style=bw
]{java}
@Test
public void userTest() throws Exception {
    |\underline{assertEquals(\textquotedbl\textbackslash{}uD83D\textbackslash{}uDE30\textquotedbl,\ StringEscapeUtils.escapeCsv(\textquotedbl\textbackslash{}uD83D\textbackslash{}uDE30\textquotedbl));}|
}
\end{minted}
\vspace*{-6pt}
\subcaption{User Triggering Test, extracted from the bug report.}
\label{fig:motivating-user}
\end{minipage}

\bigskip

\begin{minipage}{\linewidth}
{
\begin{minted}[escapeinside=||,linenos,
  firstnumber=249,
  fontsize=\scriptsize,
  style=bw,
  numbersep=7pt
]{java}
@Test
public void testLang857() throws Exception {
    |\underline{assertEquals(\textquotedbl\textbackslash{}uD83D\textbackslash{}uDE30\textquotedbl,\ StringEscapeUtils.escapeCsv(\textquotedbl\textbackslash{}uD83D\textbackslash{}uDE30\textquotedbl));}|
    // Examples from https://en.wikipedia.org/wiki/UTF-16
    assertEquals("\uD800\uDC00", StringEscapeUtils.escapeCsv("\uD800\uDC00"));
    assertEquals("\uD834\uDD1E", StringEscapeUtils.escapeCsv("\uD834\uDD1E"));
    assertEquals("\uDBFF\uDFFD", StringEscapeUtils.escapeCsv("\uDBFF\uDFFD"));
}
\end{minted}
}
\vspace*{-6pt}
\subcaption{Dev Triggering Test.}
\label{fig:motivating-dev}
\end{minipage}

\bigskip

\begin{minipage}{\linewidth}
{
\renewcommand\theFancyVerbLine{%
\ifnum\value{FancyVerbLine}=467
  \setcounter{FancyVerbLine}{475}\rmfamily \tiny \vdots
\else
\rmfamily \tiny \arabic{FancyVerbLine}%
\fi
}
\begin{minted}[
  linenos,
  firstnumber=466,
  fontsize=\scriptsize,
  style=borland,
  numbersep=7pt
]{diff}
public abstract class CharSequenceTranslator {
        for (int pt = 0; pt < consumed; pt++) {
-           pos += Character.charCount(Character.codePointAt(input, pt));
+           pos += Character.charCount(Character.codePointAt(input, pos));
        }
    }
}
\end{minted}
}
\vspace*{-6pt}
\subcaption{The committed bug fix.}
\label{fig:motivating-fix}
\end{minipage}
\end{figure}



\bibliographystyle{ACM-Reference-Format}
\bibliography{references}


\end{document}
